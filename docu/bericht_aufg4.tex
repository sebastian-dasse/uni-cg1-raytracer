\documentclass[11pt]{amsart}
\usepackage{ngerman}
\usepackage{geometry}                % See geometry.pdf to learn the layout options. There are lots.
\geometry{a4paper}             

%\usepackage[parfill]{parskip}    % Activate to begin paragraphs with an empty line rather than an indent
\usepackage{graphicx}
\usepackage{amssymb}
\usepackage{epstopdf}
\DeclareGraphicsRule{.tif}{png}{.png}{`convert #1 `dirname #1`/`basename #1 .tif`.png}
\title{CGI: Bericht zu Aufgabe 4, Raytracer}
\author{von Sebastian Dass\'{e}, Max Novichkov, Simon Lischka }
\date{13.01.2014}

\begin{document}

\maketitle

\section{Aufgabenstellung}
Die Implementierung der Beleuchtung sollte um Schatten, Reflexion und Refraktion erweitert werden. Dazu gab es folgende 
Teilaufgaben: 

\subsection{}
Der Welt sollte ein Attribut f\"ur den Brechungsindex hinzugef\"ugt werden.

\subsection{}
Die Klassen f\"ur das Licht sollten ein zus\"atzliches Attribut \texttt{castsShadow} erhalten, das an den Konstruktor 
als Parameter \"ubergeben wird und das bestimmt, ob das entsprechende Licht einen Schatten wirft. Au\ss{}erdem sollte 
in der Methode \texttt{illuminates} die \texttt{World} als Parameter \"ubergeben werden.

\subsection{}
Es war eine Klasse \texttt{Tracer} zu implementieren, die eine Methode zum Raytracen liefert. Die Klassen f\"ur das 
Material waren so zu ver\"andern, dass in der Methode \texttt{colorFor} der \texttt{Tracer} als Parameter mit 
\"ubergeben wird.

\subsection{}
Zwei neue Materialien sollten implementiert werden: \texttt{ReflectiveMaterial}, das Material f\"ur einen perfekt 
diffus reflektiernden K\"orper mit Glanzpunkt und Reflexion, und \texttt{TransparentMaterial}, das Material f\"ur einen 
perfekt transparenten K\"orper.

\subsection{}
Nach all diesen Modifikationen waren drei vorgegebene Szenen zu rendern.



\section{L\"osungsstrategien}
\subsection{}
Zun\"achst wurden alle rein "`formalen"' \"Anderungen der Methodenheader usw. umgesetzt und die neuen Klassen nach 
bew\"ahrtem Schema als leere Vorlagen generiert.

\subsection{}
Die aus der Vorlesung bekannten Formeln wurden nach M\"oglichkeit an passender Stelle direkt in Code umgesetzt und das 
Ergebnis an kleinen Testszenen ausprobiert.



\section{Besondere Probleme}
\subsection{}
Beim Schatten traten, wie zu erwarten war, anfangs Artefakte auf. Die Schwierigkeit dabei war nun, "`an der richtigen 
Stelle"' den Vergleich von t mit 0 durch einen Vergleich von t mit einem kleinen positiven $\varepsilon$ zu ersetzen.

\subsection{}
Die Umsetzung der Refraktion erwies sich als erhebliche Schwierigkeit. Besondere Probleme bereitete die Frage, wie bei 
der \texttt{AxisAlignedBox} f\"ur den Fall, dass man sich "`innen"' befindet, die "`richtige"', d.h. die vom Strahl 
getroffene Seite herausgesucht werden sollte.



\section{Implementierung}
\subsection{Schatten}
Bei allen nicht abstrakten Klassen f\"ur das Licht wird in der Methode \texttt{illuminates} gepr\"uft, ob 
\texttt{castsShadow == false}. In diesem Fall wird direkt \texttt{true} zur\"ueckgegeben. Andernfalls wird nun 
gepr\"uft, ob sich ein Objekt zwischen dem untersuchten Punkt und der Lichtquelle befindet, bzw. beim 
\texttt{DirectionalLight}: ob sich \"uberhaupt ein Objekt auf dem Strahl vom Punkt in Richtung Licht befindet.

In allen Klassen f\"ur das Material wurde nochmals gepr\"uft, dass ein Licht nur zur Beleuchtung des jeweiligen Punkts 
beitr\"agt, wenn die entsprechende Methode \texttt{illuminates true} zur\"uckgibt.


\subsection{Reflexion und Refraktion}
Reflexion und Refraktion wurden nach den Formeln, die in der Aufgabe vorgegeben wurden, implementiert.



\section{Zeitbedarf}
\subsection{}
Der Zeitbedarf wurde bei allen Teammitgliedern mit jeweils 20 Stunden angegeben.


\end{document} 
