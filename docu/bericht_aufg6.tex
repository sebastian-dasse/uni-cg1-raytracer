\documentclass[11pt]{amsart}
\usepackage{ngerman}
\usepackage{geometry}                % See geometry.pdf to learn the layout options. There are lots.
\geometry{a4paper}             

%\usepackage[parfill]{parskip}    % Activate to begin paragraphs with an empty line rather than an indent
\usepackage{graphicx}
\usepackage{amssymb}
\usepackage{epstopdf}
\DeclareGraphicsRule{.tif}{png}{.png}{`convert #1 `dirname #1`/`basename #1 .tif`.png}
\title{CGI: Bericht zu Aufgabe 6, Raytracer}
\author{von Sebastian Dass\'{e}, Max Novichkov, Simon Lischka }
\date{10.02.2014}

\begin{document}

\maketitle

\section{Aufgabenstellung}
Implementierung der Texturierung.

\subsection{}
Dazu sollte zun\"achst ein UML-Diagramm des Raytracers mit allen geplanten \"Anderungen angefertigt werden.

\subsection{}
Dann sollte im ersten Schritt eine Klasse \texttt{TexCoord2} f\"ur die Texturkoordinaten  und die Berechnung der 
Texturkoordinaten f\"ur die Geometrieklassen implementiert werden.

\subsection{}
Nun sollten drei verschiedene Texturklassen implementiert werden, die alle auf dem zu definierenden Interface 
\texttt{Texture} basieren sollten: \texttt{SingleColorTexture} f\"ur einfarbige Oberfl\"achen, \texttt{ImageTexture} 
f\"ur eine Textur auf der Grundlage eines Bildes und \texttt{InterpolatedImageTexture} f\"ur eine Textur mit einem 
bilinear interpolierten, im Original gering aufgel\"osten Bild.

\subsection{}
Zus\"atzlich sollte ein Material implementiert werden, dass in Anh\"angigkeit von der St\"arke der Beleuchtung eines 
Punktes eine von zwei alternativen Texturen anzeigt.

\subsection{}
Nach all diesen Modifikationen waren vier vorgegebene Szenen zu rendern. Au\ss{}erdem sollte unter Verwendung der 
meisten implementierten Techniken eine freie Szene erstellt werden.



\section{L\"osungsstrategien}
\subsection{}
Zun\"achst wurden alle rein "`formalen"' \"Anderungen der Methodenheader usw. umgesetzt und die neuen Klassen nach 
bew\"ahrtem Schema als leere Vorlagen generiert.

\subsection{}
Die aus der Vorlesung bekannten Formeln wurden nach M\"oglichkeit an passender Stelle direkt in Code umgesetzt und das 
Ergebnis an kleinen Testszenen ausprobiert.

\subsection{}
Das UML-Klassendiagramm erwies sich als hilfreiches Werkzeug zur Planung und um den \"Uberblick \"uber die 
Zusammenh\"ange zwischen den Klassen im Raytracer zu behalten.



\section{Besondere Probleme}
\subsection{}
Bei der \texttt{ImageTexture} dauerte das Rendering bei der ersten Implementierung aussergew\"ohnlich lange. Das 
lie\ss{} sich dadurch l\"osen, dass in der \texttt{getColor}-Methode die Breite und H\"ohe des Bildes nicht mehr \"uber 
Methodenaufrufe an das \texttt{BufferedImage} ermittelt wurde sondern, sich diese Werte in Instanzvariablen gemerkt 
werden. Au\ss{}erdem wurden alle Variablen \texttt{final} gesetzt.

\subsection{}
Eine weitere kleine Schwierigkeit bei der \texttt{ImageTexture} war es, die Konvertierung von Farben vom 
\texttt{BufferedImage} in unser eigenes \texttt{Color}-Format korrekt umzusetzen.



\section{Implementierung}
\subsection{Texturkoordinaten und deren Berechnung}
In der Klasse \texttt{TexCoord2} werden im Konstruktor die als Parameter \"ubergebenen Werte u und v in positive Werte 
umgewandelt und Modulo 1 gerechnet. Auf diese Weise werden alle Texturen "`gekachelt"'. \\
Die Berechnung der Texturkoordinaten in den einzelnen Geometrien wurden nach den Formeln, die in der Aufgabe vorgegeben 
wurden, implementiert.


\subsection{Texturen}
F\"ur die beiden Bild-Texturen wurde eine abstrakte Superklasse \\ \texttt{AbstractImageTexture} definiert, in der das 
eigentliche Laden des Bildes implementiert ist und Instanzvariablen f\"ur das Bild, dessen Raster und dessen Breite und 
H\"ohe existieren. \\
In der \texttt{ImageTexture} wurde die Umrechnung der Texturkoordinaten auf die Bildkoordinaten durch einfache 
Multiplikation mit der Breite und Umwandlung in ganzzahlige Werte gel\"ost. \\
In der \texttt{InterpolatedImageTexture} wurde die Koordinatenumrechnung nach den Formeln, die in der Aufgabe 
vorgegeben wurden, implementiert.


\subsection{Material f\"ur Tag und Nacht}
Das \texttt{DayAndNightMaterial} wurde so implementiert, dass im Konstruktor zwei Materialien \"ubergeben werden. In 
der \texttt{colorFor}-Methode wird zum Testen der Beleuchtungsintensit\"at f\"ur einen Punkt ein 
\texttt{LambertMaterial} mit der Farbe wei\ss{} gepr\"uft und der entprechende Farbwert in einen Helligkeitswert 
umgerechnet. Je nachdem ob dieser nun einen festgelegten Schwellenwert \"uber- oder unterschreitet wird das Material 
f\"ur den Tag oder f\"ur die Nacht angezeigt.



\section{Zeitbedarf}
\subsection{}
Der Zeitbedarf wurde bei allen Teammitgliedern mit jeweils 15 Stunden angegeben.


\end{document} 
