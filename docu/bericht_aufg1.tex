\documentclass[11pt]{amsart}
\usepackage{geometry}                % See geometry.pdf to learn the layout options. There are lots.
\geometry{a4paper}             

%\usepackage[parfill]{parskip}    % Activate to begin paragraphs with an empty line rather than an indent
\usepackage{graphicx}
\usepackage{amssymb}
\usepackage{epstopdf}
\DeclareGraphicsRule{.tif}{png}{.png}{`convert #1 `dirname #1`/`basename #1 .tif`.png}
\title{CGI: Bericht zu Aufgabe 1, Raytracer}
\author{von Sebastian Dass\'{e}, Max Novichkov, Simon Lischka }
\date{11.11.2013}    

\begin{document}

\maketitle

\section{Aufgabenstellung}
\subsection{}
Implementierung einer Klasse zum Laden und Anzeigen eines Bildes, das mit einem File-Open Dialog ge\"offnet wird.

\subsection{}
Pixelweises zeichnen einer Diagonale auf einem schwarzen Hintergrund, welches sich korrekt der Fenstergr\"o\ss{}e anpasst. Funktionalit\"at zum Speichern des Bildes als PNG-Datei.

\subsection{}
Generierung einer Matrizen- und Vektorenklasse, die spezifische Operationen zu 3er Matrizen, Vektoren, Punkten und Normalen anbietet.

\subsection{}
Testen der Matrizen- und Vektorenklasse durch Durchf\"uhrung und \"Uberpr\"ufung exemplarischer Rechnungen.


\section{L\"osungsstrategien}
\subsection{}
Zun\"achst wurde die grundlegende Klassenstruktur unter Zuhilfenahme der Java API-Doc geplant. Die Klasse wurden 
dann als leere Vorlagen generiert, um nach dem Divide-and-Conquer Prinzip eine st\"uckweise Erarbeitung des Programmes
durch die Teammitglieder zu erm\"oglichen.

\subsection{}
Verschachtelte oder voneinander abh\"anginge Objekte wurden strukturell erarbeitet und getestet, damit anschlie\ss{}end 
die korrekte Funktionsweise im Gesamtkontext hergestellt werden konnte. Fehler in dem Ausgabebild o.\"a. sind in dieser
zu ignorieren.

\subsection{}
Teaminterne Kommunikation und Informationsaustausch um Beitr\"age aller Mitglieder zu f\"ordern wurde in regelm\"a\ss{}igen Meetings 
vorgenommen.


\section{Besondere Probleme}
\subsection{}
Ein Problem entstand bei der Implementierung der ImageViewer-Klasse. Zun\"achst war es vorgesehen, das Bild
auch auf ein Canvas zu zeichnen. Allerdings war es bis zum Abgabezeitpunkt nicht m\"glich, die Gr\"o\ss{}en 
verschiedener Bilder bei jeder getesteten Datei korrekt auf das Canvas zu \"ubertragen.


\section{Implementierung}
\subsection{}
Bei der Implementierung des ImageViewers ging es darum, einen m\"oglichst effizienten L\"osungsweg zu finden. Aus diesem Grund
griffen wir auf ein ImageIcon Objekt zu, welches f\"ur die korrekte Darstellung sorgt und die Gr\"o\ss{}e  mit
dem pack() Befehl anpassen kann.

\subsection{}
Um ein Bild zu generieren und zu speichern, wurde in einer ImageCanvas-Klasse ein BufferedImage-Objekt als Instanzvariable
festgelegt, welches nach Extrahieren des Rasters zum Zeichnen in der paint()-Methode des
Canvas benutzt wird.
Mit dieser Vorgehensweise ist es m\"oglich, klassenintern das BufferedImage Objekt auszutauschen, 
um die Gr\"o\ss{}e des Bildes z.b. bei Ver\"anderungen der Fenstergr\"o\ss{}e anzupassen.

\subsection{}
Die Vektor- und Matritzenklassen wurden gem\"a\ss{} den Vorgaben implementiert und werfen IllegalArgumentExceptions, 
wenn die \"ubergebenen Parameter falsch sind. Dies wird vor dem Durchf\"uhren der Kalkulation gepr\"uft.
Um ein zuverl\"assiges Testen der Funktionen zu Erm\"oglichen, wurde zudem ein getrenntes Testpackage angelegt,
in dem die Tests mit JUnit durchgef\"uhrt werden. Diese Struktur erm\"oglicht eine problemlose Expandierung der Tests
mit wachsendem Projektumfang.


\section{Zeitbedarf}
\subsection{}
Der Zeitbedarf wurde bei allen Teammitgliedern mit jeweils 10-15 Stunden angegeben. Dabei 
wurde viel Zeit damit verwendet, sich auf das GIT-System einzuarbeiten und die zugrundeliegenden
Komponenten der Bilddarstellung nachzuvollziehen.


\end{document} 
