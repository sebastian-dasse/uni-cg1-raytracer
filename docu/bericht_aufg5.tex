\documentclass[11pt]{amsart}
\usepackage{ngerman}
\usepackage{geometry}                % See geometry.pdf to learn the layout options. There are lots.
\geometry{a4paper}             

%\usepackage[parfill]{parskip}    % Activate to begin paragraphs with an empty line rather than an indent
\usepackage{graphicx}
\usepackage{amssymb}
\usepackage{epstopdf}
\DeclareGraphicsRule{.tif}{png}{.png}{`convert #1 `dirname #1`/`basename #1 .tif`.png}
\title{CGI: Bericht zu Aufgabe 5, Raytracer}
\author{von Sebastian Dass\'{e}, Maxim Novichkov, Simon Lischka }
\date{27.01.2014}

\begin{document}

\maketitle

\section{Aufgabenstellung}

\subsection{}
Zeichnung einer UML Diagramme

\subsection{}
Implementierung der 4x4 Matrizen und Transformationsklassen

\subsection{}
Implementierung einer Nodeklasse

\subsection{}
Implementierung einer Loader f\"ur das OBJ Format

\subsection{}
Notwendige Anpassungen an die Geometrieklassen

\section{L\"osungsstrategien}
\subsection{}
Die Klassen wurden zun\"achst als leere Vorlagen nach den UML Diagrammen generiert.

\subsection{}
Die Formeln der entsprechenden Klassen wurden aus der Vorlesungsmaterial genommen.

\section{Besondere Probleme}

\subsection{}
Die schwierigste Aufgabe war das korrekte Parsing der gegebenen Dateien und die entsprechende Umwandlung gewonnenen Daten in
eine Figur.   
F\"ur Testen, die Suche und Aufhebung der entstehenden Problemen wurde sehr viel Zeit in Anspruch genommen.


\section{Implementierung}
\subsection{}
Die 4x4 Matrize und Transformationsklassen wurden nach den Formeln, die im Vorlesungsmaterial angegeben 
wurden und unter Ber\"ucksichtigung des UML Diagramms implementiert.

\subsection{}
Die Nodeklasse wurde nach in der Vorlesung besprochener Konzeption entworfen. 

\subsection{}
Nach dem langen Testen wurde Loader f\"ur das OBJ Format implementiert. 

\section{Zeitbedarf}
\subsection{}
Der Zeitbedarf wurde bei allen Teammitgliedern mit jeweils 20 Stunden angegeben.



\end{document} 
