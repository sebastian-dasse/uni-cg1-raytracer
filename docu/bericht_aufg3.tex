\documentclass[11pt]{amsart}
\usepackage{geometry}                % See geometry.pdf to learn the layout options. There are lots.
\geometry{a4paper}             

%\usepackage[parfill]{parskip}    % Activate to begin paragraphs with an empty line rather than an indent
\usepackage{graphicx}
\usepackage{amssymb}
\usepackage{epstopdf}
\DeclareGraphicsRule{.tif}{png}{.png}{`convert #1 `dirname #1`/`basename #1 .tif`.png}
\title{CGI: Bericht zu Aufgabe 3, Raytracer}
\author{von Sebastian Dass\'{e}, Max Novichkov, Simon Lischka }
\date{9.12.2013}    

\begin{document}

\maketitle

\section{Aufgabenstellung}
\subsection{}
Implementierung der Materialklassen Lambert, Phong und SingleColor

\subsection{}
Implementierung der Lichtklassen SpotLight, PointLight und DirectionalLight

\subsection{}
Entwerfen einer eigenen Demoszene


\section{L\"osungsstrategien}
\subsection{}
Die Klassen wurden zun\"achst als leere Vorlagen generiert.

\subsection{}
Die entsprechenden Klassen wurden zun\"achst theoretisch vorbereitet und dann gemeinsam im linearen Zeitablauf implementiert.

\subsection{}
Bei unerwarteten Fehlern wurden Testausgaben mit entsprechenden Zahlenwerten, beispielsweise
von Colorwerten beim Entwerfen der Beleuchtung, vorgenommen und analysiert. Dabei lie\ss{}en wir 
uns gezielt fehlerverd\"achtige Wertebereiche anzeigen.



\section{Besondere Probleme}
\subsection{}
In der colorFor Methode der Materialien muss der zur\"uckgegebene Farbwert
kompensiert werden, um den maximalen Wertebereich von 0-1 nicht zu verletzen.
Eine Schwierigkeit bestand darin, einen geeigneten Kompensationsfaktor zu finden
(momentan Anzahl der Lichtquellen + maximaler Farbwert des Ambientlights).

\section{Implementierung}
\subsection{}
Die Materialklassen wurden nach den Formeln, die in der Aufgabe vorgegeben 
sind und unter Ber\"ucksichtigung des gegebenen Klassendiagramms implementiert.

\subsection{}
Unter der gleichen Vorgehensweise wurden die Lichtquellen implementiert.


\section{Zeitbedarf}
\subsection{}
Der Zeitbedarf wurde bei allen Teammitgliedern mit jeweils 15 Stunden angegeben. Dabei 
wurde der gr\"o{\ss}te Teil der Zeit in der Implementierung und Fehlersuche verwendet.


\end{document} 
