\documentclass[11pt]{amsart}
\usepackage{geometry}                % See geometry.pdf to learn the layout options. There are lots.
\geometry{a4paper}             

%\usepackage[parfill]{parskip}    % Activate to begin paragraphs with an empty line rather than an indent
\usepackage{graphicx}
\usepackage{amssymb}
\usepackage{epstopdf}
\DeclareGraphicsRule{.tif}{png}{.png}{`convert #1 `dirname #1`/`basename #1 .tif`.png}
\title{CGI: Bericht zu Aufgabe 2, Raytracer}
\author{von Sebastian Dass\'{e}, Max Novichkov, Simon Lischka }
\date{24.11.2013}    

\begin{document}

\maketitle

\section{Aufgabenstellung}
\subsection{}
Implementierung der beiden Kameras OrtographicCamera und PerspectiveCamera.

\subsection{}
Schreiben einer Color-Klasse.

\subsection{}
Entwerfen einer abstrakten Geometrie-Superklasse Geometry und der entsprechenden Unterklassen Plane, AxisAlignedBox, Triangle und Sphere sowie eines Hit-Objekts.
\subsection{} 
Entwerfen einer Welt-Klasse, die alle Objekte der Szene enth\"alt
\subsection{} 
Entwerfen der Raytracer-Klasse, die \"uber die Objekte der Welt iteriert.
\subsection{}
Testen der Geometry-Objekte durch entsprechende Beispielkonfigurationen.


\section{L\"osungsstrategien}
\subsection{}
Die Klassen wurden zun\"achst als leere Vorlagen generiert.

\subsection{}
Die entsprechenden Klassen wurden zun\"achst auf Papier
gerechnet und dann gemeinsam im linearen Zeitablauf implementiert.

\subsection{}
Raytracer und View Objekte wurden m\"oglichst generisch modularisiert.


\section{Besondere Probleme}
\subsection{}
Einsetzen der Werte aus der selbsterstellten Color-Klasse in das WriteableRaster.

\subsection{}
Implementierung der AxisAlignedBox.


\section{Implementierung}
\subsection{}
Die Kameras, die Klassen Color und Hit sowie die Geometrie-Klassen wurden 
anhand der im Unterricht vorgegebenen Methodik und dem Buch 
\textit{Ray Tracing From Ground} Up von Kevin Suffern
umgesetzt. 

\subsection{}
Der Raytracer gibt mit der Methode trace() ein BufferedImage-Objekt aus, das mit der UI-Klasse ShowImage auf einem Canvas dargestellt wird.
Zur Bequemen Erzeugung der Graphikobjekte wurde die Factoryklasse raytracer.tests.graphical.Factory erstellt.


\section{Zeitbedarf}
\subsection{}
Der Zeitbedarf wurde bei allen Teammitgliedern mit jeweils 25 Stunden angegeben. Dabei 
wurde viel Zeit f\"ur mathematische \"Uberlegungen aufgewandt.


\end{document} 
